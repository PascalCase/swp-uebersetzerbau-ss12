\documentclass[ucs,9pt]{beamer}
% frueher mal:
% \documentclass[hyperref={pdfpagelabels=false},ucs,9pt]{beamer}

% Copyright 2004 by Till Tantau <tantau@users.sourceforge.net>.
%
% In principle, this file can be redistributed and/or modified under
% the terms of the GNU Public License, version 2.
%
% However, this file is supposed to be a template to be modified
% for your own needs. For this reason, if you use this file as a
% template and not specifically distribute it as part of a another
% package/program, I grant the extra permission to freely copy and
% modify this file as you see fit and even to delete this copyright
% notice.
%
% Modified by Tobias G. Pfeiffer <tobias.pfeiffer@math.fu-berlin.de>
% to show usage of some features specific to the FU Berlin template.

% remove this line and the "ucs" option to the documentclass when your editor is not utf8-capable
\usepackage[utf8x]{inputenc}    % to make utf-8 input possible
\usepackage[ngerman]{babel}     % hyphenation etc., alternatively use 'german' as parameter
\usepackage{graphicx}
\usepackage{listings}
\usepackage{multirow}

% Template for talks using the Corporate Design of the Freie Universitaet
%   Berlin, created following the guidelines on www.fu-berlin.de/cd by
%   Tobias G. Pfeiffer, <tobias.pfeiffer@math.fu-berlin.de>
% This file can be redistributed and/or modified in any way you like.
%   If you feel you have done significant improvements to this template,
%   please consider providing your modified version to
%   https://www.mi.fu-berlin.de/w/Mi/BeamerTemplateCorporateDesign

\usepackage{amsmath,dsfont,listings}

%%% FU logo
% small version for upper right corner of normal pages
\pgfdeclareimage[height=0.9cm]{university-logo}{FULogo_RGB}
\logo{\pgfuseimage{university-logo}}
% large version for upper right corner of title page
\pgfdeclareimage[height=1.085cm]{big-university-logo}{FULogo_RGB}
\newcommand{\titleimage}[1]{\pgfdeclareimage[height=2.92cm]{title-image}{#1}}
\titlegraphic{\pgfuseimage{title-image}}
%%% end FU logo

% NOTE: 1cm = 0.393 in = 28.346 pt;    1 pt = 1/72 in = 0.0352 cm
\setbeamersize{text margin right=3.5mm, text margin left=7.5mm}  % text margin

% colors to be used
\definecolor{text-grey}{rgb}{0.45, 0.45, 0.45} % grey text on white background
\definecolor{bg-grey}{rgb}{0.66, 0.65, 0.60} % grey background (for white text)
\definecolor{fu-blue}{RGB}{0, 51, 102} % blue text
\definecolor{fu-green}{RGB}{153, 204, 0} % green text
\definecolor{fu-red}{RGB}{204, 0, 0} % red text (used by \alert)

% switch off the sidebars
% TODO: loading \useoutertheme{sidebar} (which is maybe wanted) also inserts
%   a sidebar on title page (unwanted), also indents the page title (unwanted?),
%   and duplicates the navigation symbols (unwanted)
\setbeamersize{sidebar width left=0cm, sidebar width right=0mm}
\setbeamertemplate{sidebar right}{}
\setbeamertemplate{sidebar left}{}
%    XOR
% \useoutertheme{sidebar}

% frame title
% is truncated before logo and splits on two lines
% if neccessary (or manually using \\)
\setbeamertemplate{frametitle}{%
    \vskip-30pt \color{text-grey}\large%
    \begin{minipage}[b][23pt]{80.5mm}%
    \flushleft\insertframetitle%
    \end{minipage}%
}

%%% title page
% TODO: get rid of the navigation symbols on the title page.
%   actually, \frame[plain] *should* remove them...
\setbeamertemplate{title page}{
% upper right: FU logo
\vskip2pt\hfill\pgfuseimage{big-university-logo} \\
\vskip6pt\hskip3pt
% title image of the presentation
\begin{minipage}{11.6cm}
\hspace{-1mm}\inserttitlegraphic
\end{minipage}

% set the title and the author
\vskip14pt
\parbox[top][1.35cm][c]{11cm}{\color{text-grey}\inserttitle \\ \small \insertsubtitle}
\vskip11pt
\parbox[top][1.35cm][c]{11cm}{\small \insertauthor \\ \insertinstitute \\[3mm] \insertdate}
}
%%% end title page

%%% colors
\usecolortheme{lily}
\setbeamercolor*{normal text}{fg=black,bg=white}
\setbeamercolor*{alerted text}{fg=fu-red}
\setbeamercolor*{example text}{fg=fu-green}
\setbeamercolor*{structure}{fg=fu-blue}

\setbeamercolor*{block title}{fg=white,bg=black!50}
\setbeamercolor*{block title alerted}{fg=white,bg=black!50}
\setbeamercolor*{block title example}{fg=white,bg=black!50}

\setbeamercolor*{block body}{bg=black!10}
\setbeamercolor*{block body alerted}{bg=black!10}
\setbeamercolor*{block body example}{bg=black!10}

\setbeamercolor{bibliography entry author}{fg=fu-blue}
% TODO: this doesn't work at all:
\setbeamercolor{bibliography entry journal}{fg=text-grey}

\setbeamercolor{item}{fg=fu-blue}
%\setbeamercolor{navigation symbols}{fg=text-grey,bg=bg-grey}
%%% end colors

%%% headline
\setbeamertemplate{headline}{
\vskip4pt\hfill\insertlogo\hspace{3.5mm} % logo on the right

\vskip6pt\color{fu-blue}\rule{\textwidth}{0.4pt} % horizontal line
}
%%% end headline

%%% footline
\newcommand{\footlinetext}{\insertshortinstitute, \insertshorttitle, \insertshortdate}
%\newcommand{\footlinetext}{\insertshorttitle, \insertshortdate}
\setbeamertemplate{footline}{
\vskip5pt\color{fu-blue}\rule{\textwidth}{0.4pt}\\ % horizontal line
\vskip2pt
\makebox[123mm]{\hspace{7.5mm}
\color{fu-blue}\footlinetext
%\hfill \raisebox{-1pt}{\usebeamertemplate***{navigation symbols}}
\hfill \insertframenumber}
\vskip4pt
}
%%% end footline

%%% settings for listings package
\lstset{extendedchars=true, showstringspaces=false, basicstyle=\footnotesize\sffamily, tabsize=2, breaklines=true, breakindent=10pt, frame=l, columns=fullflexible}
\lstset{language=Java} % this sets the syntax highlighting
\lstset{mathescape=true} % this switches on $...$ substitution in code
% enables UTF-8 in source code:
\lstset{literate={ä}{{\"a}}1 {ö}{{\"o}}1 {ü}{{\"u}}1 {Ä}{{\"A}}1 {Ö}{{\"O}}1 {Ü}{{\"U}}1 {ß}{\ss}1}
%%% end listings
  % THIS is the line that includes the FU template!
%\include{algorithmic}

\usepackage{arev,t1enc} % looks nicer than the standard sans-serif font
% if you experience problems, comment out the line above and change
% the documentclass option "9pt" to "10pt"

% image to be shown on the title page (without file extension, should be pdf or png)
\titleimage{fu_500}

\title[Softwareprojekt Übersetzerbau]
{Softwareprojekt Übersetzerbau}

\subtitle{Optimierungstechniken}

\author[Knötel, Karger, Marzin] % (optional, use only with lots of authors)
{David~Knötel, Björn~Karger, Daniel Marzin}
% - Give the names in the same order as the appear in the paper.

\institute[FU Berlin] % (optional, but mostly needed)
{Freie Universität Berlin, Institut für Informatik}
% - Keep it simple, no one is interested in your street address.

\date[20.07.2012] % (optional, should be abbreviation of conference name)
{}
% - Either use conference name or its abbreviation.
% - Not really informative to the audience, more for people (including
%   yourself) who are reading the slides online

%\subject{Verteidigung der Bachelorarbeit}
% This is only inserted into the PDF information catalog. Can be left
% out.

% you can redefine the text shown in the footline. use a combination of
% \insertshortauthor, \insertshortinstitute, \insertshorttitle, \insertshortdate, ...
\renewcommand{\footlinetext}{\insertshorttitle, \insertshortdate}

\setbeamertemplate{navigation symbols}{}

% Delete this, if you do not want the table of contents to pop up at
% the beginning of each subsection:
\AtBeginSection[] {
	\begin{frame}<beamer>{Leitfaden}
		\tableofcontents[currentsection,currentsubsection]
	\end{frame}
}

\begin{document}

\frame[plain]{\titlepage}
%\begin{frame}[plain]
%	\titlepage
%\end{frame}

\lstset{language=C,showstringspaces=false,tabsize=4,keywordstyle=\color{red},commentstyle=\color{blue},
stringstyle=\color{brown},frame=single}

% -------------------------------------------------------------------------------------------
% -------------------------------------------------------------------------------------------

\begin{frame}{Ziele}

\begin{itemize}
\item Gesucht: Sinnvolles Maß für den Abstand zweier Kurven $f:[a_1,b_1]\rightarrow \mathds{R}^n$ und $g:[a_2,b_2]\rightarrow \mathds{R}^n$
\vspace{3mm}
\item Lösung: Einführung des Fréchet-Abstandes
\end{itemize}
\end{frame}

\begin{frame}{Programmstruktur}
UML
\end{frame}

\begin{frame}{Implementierung}
\begin{itemize}
\item LLVM ist SSA
\vspace{3mm}
\item f ist Kurve mit p Punkten $p_1,\dots ,p_p$, so dass $f:[1,p]\rightarrow \mathds{R}^n$ mit $f(1) = p_1,\dots ,f(p) = p_p$ gilt und dazwischen linear interpoliert wird
\vspace{3mm}
\item g ist Kurve mit q Punkten $q_1,\dots ,q_q$, so dass $f:[1,q]\rightarrow \mathds{R}^n$ mit $g(1) = q_1,\dots ,g(q) = q_q$ gilt und dazwischen linear interpoliert wird
\end{itemize}
\end{frame}

\begin{frame}{Eingesetzte Optimierungsalgorithmen}
\begin{itemize}
\item Entscheidungsproblem: Ist der Fréchet-Abstand zweier Kurven kleiner gleich einem gegebenen $\epsilon$?
\vspace{3mm}
\item Lösung mit Hilfe des Free Space Diagram
\vspace{3mm}
\item Definition: $$F_\epsilon (f,g) = \{(s,t)\in [1,p] \times [1,q] : \|f(s) - g(t)\| \leq \epsilon \}$$
\end{itemize}
\end{frame}

\begin{frame}{Optimierungen \& Beispiele}

\begin{columns}
\begin{column}{.48\textwidth}

\begin{itemize}
\item <+-| alert@+> Constant folding
\item <+-| alert@+> Constant propagation
\item <+-| alert@+> Store/Load folding
\item <+-| alert@+> Remove common expressions
\item <+-| alert@+> Global life time analysis
\item <+-| alert@+> Strength reduction
\item <+-| alert@+> Eliminate dead registers/blocks
\end{itemize}

\end{column}%
%\hfill%
\begin{column}{.48\textwidth}
%\color{green}\rule{\linewidth}{2pt}

{\fbox{\parbox[c][2.9cm]{5cm}{
\only<1-1>{Test1}
\only<2-2>{Test2}
\only<3-3>{Test3}
\only<4-4>{Test4}
\only<5-5>{Test5}
\only<6-6>{Test6}
\only<7-7>{Test7}
}}\\
{\color{red}unoptimierter Code}\\
\vspace*{0.3cm}
\fbox{\parbox[c][2.9cm]{5cm}{
\only<1-1>{Test1}
\only<2-2>{Test2}
\only<3-3>{Test3}
\only<4-4>{Test4}
\only<5-5>{Test5}
\only<6-6>{Test6}
\only<7-7>{Test7}

}}\\
{\color{green}optimierter Code}}

\end{column}%
\end{columns}

\end{frame}

\begin{frame}{Fazit / Ausblick}
\begin{itemize}
\item Es wurden mehrere Optimierungsalgorithmen erfolgreich auf einen gegebenen LLVM-Code angewendet. Auch bei natürlichem, aus C-Code über CLANG erzeugten LLVM-Code wurden erfolgreich bedeutende Mengen an Codezeilen entfernt.
\vspace{3mm}
\item Eine Weiterentwicklung des Programms wäre durch die Vielzahl an potentiellen weiteren Optimierungstechniken problemlos möglich. Priorität hätte hierbei die Anwendung von Schleifenoptimierungen.
\end{itemize}
\end{frame}

\begin{frame}{Danke}
	\begin{center}
		\begin{Large}Vielen Dank für Ihre Aufmerksamkeit.\end{Large}
	\end{center}
\end{frame}

% -------------------------------------------------------------------------------------------
% -------------------------------------------------------------------------------------------

\begin{frame}{Quellen}
	\begin{itemize}
		\item Aho, Alfred V. ; Lam, Monica S. ; Sethi, Ravi; Ullman, Jeffrey D.: Compilers: Principles, Techniques and Tools.
		\vspace{2mm}
		\item LLVM Language Reference Manual. http://llvm.org/docs/LangRef.html (Abruf 17.07.2012).
	\end{itemize}
\end{frame}

\end{document}
